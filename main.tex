\documentclass[letterpaper,10pt]{article}
\usepackage[utf8]{inputenc}
\usepackage[spanish]{babel}
\selectlanguage{spanish}

\title{Mecanismo programable para niños}
\author{Aldama Pérez Cristopher\newline Zavala Ventura Miguel Ángel }
\date{2015}

\begin{document}
\maketitle
\pagenumbering{gobble}
\newpage
\pagenumbering{arabic}

\section{Resumen}

Mecanismo programable para niños es un sistema que consta de un lenguaje gráfico 
de programación y un bloque micontrolador al que se le pueden conectar sensores 
(contacto, luz y temperatura) y actuadores (motores), que sirve como apoyo a la 
enseñanza de programación y robótica a niños de primaria de entre 7 y 11 años.\newline

\textbf{Palabras clave}: Lenguaje, Sensor, Actuador, Robótica, Programación, Educación.

\newpage
\section{Advertencia}
\textit{``Este trabajo contiene información desarrollada por la Escuela
 Superior de Cómputo del Instituto Politécnico Nacional a partir de datos y documentos
  con derecho de propiedad y por lo tanto su uso queda restringido a las 
  aplicaciones que explícitamente se convengan.''}

\newpage  
\tableofcontents
\newpage
  
\newpage
\section{Objetivo}

Analizar, implementar y diseñar un sistema de cómputo, tanto en hardware como
en software que permita a niños de entre 7 y 11 años crear programas simples
usando iconos gráficos, así como su ejecución e interacción con sensores y motores,
con la finalidad de ayudar en la enseñanza de programación y uso de computadoras.

\newpage
\section{Introducción}

En el campo de la educación básica, tanto en escuelas públicas como privadas, 
una de las principales preocupaciones es enseñar conceptos relacionados 
con la tecnología, debido a la exposición que la sociedad tiene 
con ella es cada vez mayor, y para ello desarrollan competencias en las 
que motivan el conocimiento, uso y aplicación de la computadora en las tareas 
de la vida diaria; sin embargo, aunque en el mercado existen diversos materiales 
para su enseñanza, es difícil encontrar alguno que mantenga el interés de los niños 
pequeños y se ajuste al ritmo en el que absorben las ideas.
\newline

Actualmente los dispositivos en los que se apoya la enseñanza de estos 
conceptos son circuitos básicos, que están listos para armarse, sin embargo 
limitan la interacción a la observación de su funcionamiento, 
lo cual pierde trascendencia e interés al poco tiempo. También existen 
sistemas más robustos, mecanismos controlados por un programa de computadora, 
para el que se necesita un nivel de abstracción mayor, 
pues requiere de la comprensión de conceptos de matemáticas y lógica.
\newline

La propuesta de este proyecto consiste en crear un dispositivo cuyo 
funcionamiento pueda ser aprendido de forma gradual 
haciendo uso del juego y para ello se debe analizar, 
diseñar, probar e implementar un sistema mecánico programable que sirva como 
material auxiliar en la enseñanza de conceptos, en el área de la lógica
y la programación de sistemas de cómputo, enfocado en niños de escuelas primarias,
con edad de entre 7 y 11 años de edad.
\newpage

\section{Análisis}
En este capítulo se plantea el problema en el que este proyecto se enfoca, así como
determinar los objetivos específicos, la justificación y se en listan los resultados esperados.
\newline

\subsection{Planteamiento del problema}

La tecnología va adquiriendo día a día un lugar más importante en el
desempeño de las tareas diarias, que van desde las compras en el súper mercado,
operaciones bancarias, entretenimiento y actividades lúdicas. La computadora 
y sus aplicaciones tienen un rol central en el desarrollo de la sociedad, 
es por eso que escuelas, en especial las de educación primaria
buscan herramientas que ayuden a sus alumnos a tener un conocimiento adecuado 
sobre las ciencias de la computación, que sirvan como base en el desarrollo
del individuo.
\newline

Este trabajo terminal, se presenta como una herramienta para la educación primaria 
que los maestros pueden aprovechar para facilitar la enseñanza de conceptos
elementales sobre el funcionamiento interno de las computadoras así como 
despertar el interés de los alumnos en la programación de computadoras.
Por medio de un lenguaje de programación simple pero que toca los aspectos 
básicas de programación y hardware en forma de sensores y motores que pueden
ser usados para armar mecanismos.

\subsection{Objetivo}

A continuación se listan el objetivo general del proyecto así como los 
objetivos específicos.

\subsubsection{Objetivo General}
Analizar, implementar y diseñar un sistema de cómputo, tanto en hardware como
en software que permita a niños de entre 7 y 11 años en escuelas de educación
primaria crear programas simples
usando iconos gráficos, así como su ejecución e interacción con sensores y motores
en una computadora de bajo costo que actué como controlador principal, 
con la finalidad de ayudar en la enseñanza de programación y uso de computadoras 
en general.

\subsubsection{Objetivos Específicos}

\begin{itemize}
 \item Crear un lenguaje de programación gráfico, Olinki a partir de ahora.
 \item Crear un entorno de desarrollo integrado (EDI) con soporte para Olinki.
 \item Implementar un interprete del lenguaje de programación Olinki.
 \item Diseñar circuitos electrónicos que den soporte a los sensores de iluminación
 , contacto y temperatura, así como a los motores eléctricos.
 \item Motivar al alumno mediante el uso estímulos visuales.
 \item Definir ejemplos que muestren las capacidades del lenguaje de programación.
 \item Diseñar una carcasa que proteja los circuitos, así como el controlador 
 principal.
 \item Realizar pruebas automatizadas que muestren fallas en el diseño del lenguaje 
 de programación.
\end{itemize}

\subsection{Justificación}

Con la integración de las computadoras a nuestra vida diaria en forma de teléfonos celulares,
 relojes inteligentes, consolas de vídeo juegos, tabletas, computadoras personales, 
 etcétera, ha surgido una corriente 
que propone la enseñanza de programación en escuelas de educación primaria como apoyo
en el entendimiento de la manera en que funcionan las computadoras y sus aplicaciones, y que ha
sido adoptada en países como Estonia (2012) e Inglaterra (2014), y otros que están 
haciendo planes para incluirla en su plan de estudios como 
Finlandia, EUA, Singapur, Dinámarca, Isreal y Australia. El objetivo de exponer a los niños
al uso de computadoras tan pronto como sea posible y desarrollar habilidades técnicas en ellos,
es el de prepararlos en el mundo tecnológico en el que viven inmersos, además
 de alimentar su curiosidad en el área de ciencias de la computación con la meta de 
 satisfacer la demanda de profesionales en el área.\newline 

Por ello se toma en cuenta que el acercamiento a la programación por niños de primaria, 
requiere de herramientas adecuadas,
que simplifiquen el proceso de crear y usar un algoritmo para resolver un problema en específico 
así pues se buscan lenguajes de programación amigables con los niños pequeños como son:
Alice, Scratch, Turtle entre otros. Lenguajes que fueron creados para la educación y que 
hacen uso de elementos gráficos para la creación de programas simples así como colores llamativos,
sentencias simples, animaciones, etcétera, pero que sin embargo solo están disponibles en el
idioma inglés o que no tienen manera de interactuar directamente con hardware.\newline

Por otro lado, en nuestro país la Reforma Integral de Educación Básica (RIEB) anima
a los docentes a hacer uso de
la tecnología con la finalidad de reforzar las clases y consolidar los conocimientos adquiridos,
enfocándose en las competencias de los alumnos, las estrategias tomadas por esta reforma son las
de capacitar a los docentes en el uso de recursos multimedia, de medios de comunicación, 
el internet y creación de infraestructura como enciclomedia. Sin embargo no se hace mención 
de la enseñanza de temas o materias de las ciencias de la computación en las aulas, 
de tal manera que la 
tecnología puede ser usada como apoyo complementario en la enseñanza de las materias y cursos 
(aprender con tecnología) o como modelo pedagógico (aprender de la tecnología).\newline

Se considera de gran importancia la elaboración de este proyecto ya que propone la realización
de un lenguaje de programación simplificado, con elementos gráficos y en español el cual
puede ser usado como herramienta en la enseñanza de los conceptos clave de las ciencias de la 
computación como son: creación de algoritmos y programación de computadoras, además de la 
experimentación incitando al usuario a diseñar, armar y mejorar sus propios diseños de software
y hardware. Que complemente el modelo propuesto por la RIEB, poniendo como actor a la tecnología
en este caso la computadora y sus aplicaciones.\newline
El juego cobra vida cuando se combina la creación del programa con el ensamblado 
pieza por pieza del mecanismo, permitiendo llevar a cabo las ideas que se 
desarrollaron en la imaginación del niño, motivándolo a mejorar y  superar lo logrado
anteriormente.



\subsection{Productos o Resultados Esperados}

Al concluir este proyecto, se espera tener un intérprete del lenguaje de programación
propuesto (Olinki), un entorno de desarrollo integrado(EDI)
con soporte gráfico para el lenguaje de programación Olinki, el cual permita la creación
de programas de computadora de manera lúdica y simple mediante el uso de iconos.

Además de una plataforma de hardware basado en la raspberry pi donde se podrá programar y utilizar el EDI Olinki 
sin necesidad de una computadora externa, así como un sensor de temperatura, luminosidad, y contacto
 conectables a ésta.\newline
 De igual manera, contará con un manual de usuario donde vendrá descrito el lenguaje de programación,
 el uso del EDI, así como dos programas de ejemplo con su respectiva contra parte en hardware.
 

\section{Marco Teórico}

Para fundamentar el desarrollo y la dirección del proyecto fue necesario indagar conceptos
teóricos que le diesen forma y contexto funcional, por lo que se realizó 
investigación para conocer el ambiente de enseñanza en el que se desarrollan los niños.
Se produjeron una serie de entrevistas con una especialista en pedagogía para
aclarar dudas concretas con  la comprensión de términos, ideas y conceptos pedagógicos.
Para reafirmar lo anterior con experiencias prácticas se procedió a buscar el contacto e
interacción con niños de escuelas para conocer el posible ambiente en el que probar el
mecanismo Olinki.\newline


\subsection{Metodología Pedagógica}

Como apoyo pedagógico se tomaron en cuenta las teorias contructivista y
construccionista, ambas centran al aprendizaje como un  proceso dinámico que el individuo
debe construir activamente, en el caso de la primer teoría, el constructivismo
cuyos principales autores fueron Jean Piaget y Lev Vygotski, postula la
necesidad de dar herramientas al alumno  que le permitan desarrollar
una solución a un problema o situación presentada, tiene como base la idea de
que el conocimiento se escala, es decir que el conocimiento o experiencia
adquirida por el sujeto, servirá como andamio para la contrucción de nuevo
conocimiento apoyado en la interacción social o con el medio (aprender haciendo), 
así la persona que aprende queda como actor principal de su propio
aprendizaje. Por otro lado, el construccionismo parte de las ideas del
constructivismo llamando artefactos a las construcciones mentales del individuo
que aprende, así la realización de  actividades como en el caso de
este proyecto: la escritura de un programa y el armado de circuitos básicos,
cumplen la función de facilitadores del aprendizaje, es de notar que el
construccionismo desarrollado por Seymour Papert  da mayor peso a la parte social, dando gran importancia a
las relaciones interpersonales en el uso de material para desarrollar
conocimiento.\newline
Los planteamientos anteriormente descritos se tomaron en cuenta en la elección de estas
corrientes ya que se basan en un sistema social. Hacen referencia a que en una sociedad
la retroalimentación entre sus personajes es obligada porque la comunicación entre 
sus individuos es la base para su desarrollo; dentro de uno de los conceptos que componen 
a esta tendencia, está la definición del establecimiento de las reglas de las 
dinámicas de interacción por el conjunto de sus individuos; otro idea que la 
compone es que en la realidad de un sujeto, su sistema de acciones es
resultado de las interacciones que tiene con la sociedad en la que se desenvuelve. 
Como vemos ahora, el paradigma socio-construccionismo antecede la 
relación entre personas a la individualidad y es así como cada sujeto modela su 
realidad. De este modo podemos analizar el aprendizaje significativo de un
sujeto cuando a este se le coloca en un entorno social con el que pueda interactuar
con otros individuos, con reglas a consensar o preestablecidas, para comenzar
a experimentar la construcción de conocimiento.


\subsection{Perfiles infantiles}
\subsubsection{Segundo grado de primaria}

En el momento en que los niños llegan a esta edad (siente años aproximadamente) 
buscan la aceptación de los adultos.  Algunos 
demandarán atención de su profesor y se puede ver afectado si no se siente especial. 
Es importante fortalecer habilidades emocionales como:
\begin{itemize}
	\item Auto conocimiento
	\item Control de emociones e identificación de emociones
	\item Fortaleza emocional
	\item Determinación
	\item Autocontrol
	\item Auto estima
\end{itemize}
Mientras desarrollan habilidades intelectuales como:
\begin{itemize}
	\item Rastreo Visual
	\item Ubicación espacial
	\item Clasificación y comparación
	\item Identificación de patrones y secuencias
	\item Figuras en espejo
	\item Ubicación temporo / espacial
	\item Comparación y asociación de objetos y analogías
	\item Situaciones de la vida cotidiana
	\item Transferencia del conocimiento
	\item Inferencias

\end{itemize}
\subsubsection{Tercer y cuarto año de Primaria}

 En esta etapa suelen volverse pueden crueles con los comentarios que hacen el uno 
 al otro. Se deben orientara ser objetivos en el tema de las amistades
 y generar las habilidades necesarias para lograr relaciones de ganar ganar en 
 todo momento.
Es importante fortalecer habilidades emocionales como:
\begin{itemize}
	\item Auto conciencia
	\item Planeación
	\item Organización
	\item Inteligencia emocional
	\item Autonomía
\end{itemize}
Mientras desarrollan habilidades intelectuales como:
\begin{itemize}
	\item Clasificación y comparación
	\item Descripción
	\item Interpretación
	\item Comparación y asociación de objetos y analogías
	\item Situaciones de la vida cotidiana
	\item Transferencia del conocimiento
	\item Inferencias
	\item Análisis
	\item Lateralidad
	\item Observación
	\item Retención

\end{itemize}

\subsubsection{Quinto año de primaria}

Algunos niños pueden sentirse presionados a experimentar. 
Los niños pasan por una gran cantidad de estrés emocional generado por el entorno, 
las cuestiones de popularidad y cuestiones personales. 
Habilidades emocionales:

\begin{itemize}
	\item Auto conciencia
	\item Empatía
	\item Asertividad
	\item Inteligencia emocional
	\item Autonomía
	\item Comunicación
	\item Solución eficaz de conflictos
	\item Establecimiento de metas
\end{itemize}

e intelectuales:
\begin{itemize}
	\item Razonamiento lógico
	\item Decodificación e interpretación
	\item Identificación y selección de información
	\item Inferencia
	\item Deducción
	\item Metacognición (Capacidad del individuo para trascender y re-apilicar su propio conocimiento)
\end{itemize}

La información de los perfiles de comportamiento de los niños en diferentes edades, 
en conjunto con la información obtenida en la investigación y la aclaración de 
conceptos en la retroalimentación pedágogica, sirvió como punto de partida para saber 
qué comportamientos observar en los niños al momento de visitar escuelas. 
De manera simultánea, brindá palabras clave para el diseño de la interfaz y la 
posible retroalimentación que el sistema Olinki le dará al usuario final; 
así mismo, la aclaración de los conceptos y paradigmas pedagógicos como el
construccionismo puede ser usado como guía en el diseño del mecanismo olinki 
encajando en las ideas de ésta corriente pedagógica.

\subsection{Retroalimentación en escuelas}

El proyecto está enfocado principalmente a niños que oscilan entre los 7 y 11 años, 
lo que implica que se debe conocer cómo es el ambiente en el que se desenvuelven 
para afinar detalles con el diseño del sistema Olinki. 
Para esto era necesario convivir con grupos de niños de diferentes edades en un 
ambiente en el que estén aprendiendo de manera habitual, 
entonces se buscó un espacio en diferentes escuelas dónde se permitiera poder
observar a alumnos de segundo a quinto año.
\newline
Las escuelas que se visitaron son el instituto Atenea y el Colegio Alamilla Americano. 
En ellas se  permitió acceso a los salones de computación para presenciar las 
clases que les imparten a los alumnos de segundo a quinto grado. 
En esta experiencia se observó el tipo de programas con los que practican los niños 
el uso de la computadora, cómo son las gráficas de los programas que utilizan para 
las dinámicas, los conceptos que conocen y ocupan dentro de lo que conocen de la 
computación; además se conoció el tipo de lenguaje que ocupan las maestras para
comunicarse con los alumnos.





\end{document}
