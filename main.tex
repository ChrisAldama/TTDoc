\documentclass[letterpaper,10pt]{article}
\usepackage[utf8]{inputenc}
\usepackage[spanish]{babel}
\selectlanguage{spanish}

\title{Mecanismo programable para niños}
\author{Aldama Pérez Cristopher\\ Zavala Ventura Miguel Ángel }
\date{2015}

\begin{document}
\maketitle
\pagenumbering{gobble}
\newpage
\pagenumbering{arabic}

\section{Resumen}

Mecanismo programable para niños es un sistema que consta de un lenguaje gráfico 
de programación y un bloque micontrolador al que se le pueden conectar sensores 
(contacto, luz y temperatura) y actuadores (motores), que sirve como apoyo a la 
enseñanza de programación y robótica a niños de primaria de entre 7 y 11 años.\\

\textbf{Palabras clave}: Lenguaje, Sensor, Actuador, Robótica, Programación, Educación.

\newpage
\section{Advertencia}
\textit{``Este trabajo contiene información desarrollada por la Escuela
 Superior de Cómputo del Instituto Politécnico Nacional a partir de datos y documentos
  con derecho de propiedad y por lo tanto su uso queda restringido a las 
  aplicaciones que explícitamente se convengan.''}

\newpage  
\tableofcontents
\newpage
  
\newpage
\section{Objetivo}

Analizar, implementar y diseñar un sistema de cómputo, tanto en hardware como
en software que permita a niños de entre 7 y 11 años crear programas simples
usando iconos gráficos, así como su ejecución e interacción con sensores y motores,
con la finalidad de ayudar en la enseñanza de programación y uso de computadoras.

\newpage
\section{Introducción}

En el campo de la educación básica, tanto en escuelas públicas como privadas, 
una de las principales preocupaciones es enseñar conceptos relacionados 
con la tecnología, debido a la exposición que la sociedad tiene 
con ella es cada vez mayor, y para ello desarrollan competencias en las 
que motivan el conocimiento, uso y aplicación de la computadora en las tareas 
de la vida diaria; sin embargo, aunque en el mercado existen diversos materiales 
para su enseñanza, es difícil encontrar alguno que mantenga el interés de los niños 
pequeños y se ajuste al ritmo en el que absorben las ideas.
\newline

Actualmente los dispositivos en los que se apoya la enseñanza de estos 
conceptos son circuitos básicos, que están listos para armarse, sin embargo 
limitan la interacción a la observación de su funcionamiento, 
lo cual pierde trascendencia e interés al poco tiempo. También existen 
sistemas más robustos, mecanismos controlados por un programa de computadora, 
para el que se necesita un nivel de abstracción mayor, 
pues requiere de la comprensión de conceptos de matemáticas y lógica.
\newline

La propuesta de este proyecto consiste en crear un dispositivo cuyo 
funcionamiento pueda ser aprendido de forma gradual 
haciendo uso del juego y para ello se debe analizar, 
diseñar, probar e implementar un sistema mecánico programable que sirva como 
material auxiliar en la enseñanza de conceptos, en el área de la lógica
y la programación de sistemas de cómputo, enfocado en niños de escuelas primarias,
con edad de entre 7 y 11 años de edad.
\newpage

\section{Análisis}
En este capítulo se plantea el problema en el que este proyecto se enfoca, así como
determinar los objetivos específicos, la justificación y se en listan los resultados esperados.
\newline

\subsection{Planteamiento del problema}

La tecnología va adquiriendo día a día un lugar más importante en el
desempeño de las tareas diarias, que van desde las compras en el súper mercado,
operaciones bancarias, entretenimiento y actividades lúdicas. La computadora 
y sus aplicaciones tienen un rol central en el desarrollo de la sociedad, 
es por eso que escuelas, en especial las de educación primaria
buscan herramientas que ayuden a sus alumnos a tener un conocimiento adecuado 
sobre las ciencias de la computación, que sirvan como base en el desarrollo
del individuo.
\newline

Este trabajo terminal, se presenta como una herramienta para la educación primaria 
que los maestros pueden aprovechar para facilitar la enseñanza de conceptos
elementales sobre el funcionamiento interno de las computadoras así como 
despertar el interés de los alumnos en la programación de computadoras.
Por medio de un lenguaje de programación simple pero que toca los aspectos 
básicas de programación y hardware en forma de sensores y motores que pueden
ser usados para armar mecanismos.

\subsection{Objetivo}

A continuación se listan el objetivo general del proyecto así como los 
objetivos específicos.

\subsubsection{Objetivo General}
Analizar, implementar y diseñar un sistema de cómputo, tanto en hardware como
en software que permita a niños de entre 7 y 11 años en escuelas de educación
primaria crear programas simples
usando iconos gráficos, así como su ejecución e interacción con sensores y motores
en una computadora de bajo costo que actué como controlador principal, 
con la finalidad de ayudar en la enseñanza de programación y uso de computadoras 
en general.

\subsubsection{Objetivos Específicos}

\begin{itemize}
 \item Crear un lenguaje de programación gráfico, Olinki a partir de ahora.
 \item Crear un entorno de desarrollo integrado (EDI) con soporte para Olinki.
 \item Implementar un interprete del lenguaje de programación Olinki.
 \item Diseñar circuitos electrónicos que den soporte a los sensores de iluminación
 , contacto y temperatura, así como a los motores eléctricos.
 \item Motivar al alumno mediante el uso estímulos visuales.
 \item Definir ejemplos que muestren las capacidades del lenguaje de programación.
 \item Diseñar una carcasa que proteja los circuitos, así como el controlador 
 principal.
 \item Realizar pruebas automatizadas que muestren fallas en el diseño del lenguaje 
 de programación.
\end{itemize}

\subsection{Justificación}

Con la integración de las computadoras a nuestra vida diaria en forma de teléfonos celulares,
 relojes inteligentes, consolas de vídeo juegos, tabletas, computadoras personales, 
 etcétera, ha surgido una corriente 
que propone la enseñanza de programación en escuelas de educación primaria como apoyo
en el entendimiento de la manera en que funcionan las computadoras y sus aplicaciones, y que ha
sido adoptada en países como Estonia (2012) e Inglaterra (2014), y otros que están 
haciendo planes para incluirla en su plan de estudios como 
Finlandia, EUA, Singapur, Dinámarca, Isreal y Australia. El objetivo de exponer a los niños
al uso de computadoras tan pronto como sea posible y desarrollar habilidades técnicas en ellos,
es el de prepararlos en el mundo tecnológico en el que viven inmersos, además
 de alimentar su curiosidad en el área de ciencias de la computación con la meta de 
 satisfacer la demanda de profesionales en el área.\\ 

Por ello se toma en cuenta que el acercamiento a la programación por niños de primaria, 
requiere de herramientas adecuadas,
que simplifiquen el proceso de crear y usar un algoritmo para resolver un problema en específico 
así pues se buscan lenguajes de programación amigables con los niños pequeños como son:
Alice, Scratch, Turtle entre otros. Lenguajes que fueron creados para la educación y que 
hacen uso de elementos gráficos para la creación de programas simples así como colores llamativos,
sentencias simples, animaciones, etcétera, pero que sin embargo solo están disponibles en el
idioma inglés o que no tienen manera de interactuar directamente con hardware.\\\\

Por otro lado, en nuestro país la Reforma Integral de Educación Básica (RIEB) anima
a los docentes a hacer uso de
la tecnología con la finalidad de reforzar las clases y consolidar los conocimientos adquiridos,
enfocándose en las competencias de los alumnos, las estrategias tomadas por esta reforma son las
de capacitar a los docentes en el uso de recursos multimedia, de medios de comunicación, 
el internet y creación de infraestructura como enciclomedia. Sin embargo no se hace mención 
de la enseñanza de temas o materias de las ciencias de la computación en las aulas, 
de tal manera que la 
tecnología puede ser usada como apoyo complementario en la enseñanza de las materias y cursos 
(aprender con tecnología) o como modelo pedagógico (aprender de la tecnología).\\\\

Se considera de gran importancia la elaboración de este proyecto ya que propone la realización
de un lenguaje de programación simplificado, con elementos gráficos y en español el cual
puede ser usado como herramienta en la enseñanza de los conceptos clave de las ciencias de la 
computación como son: creación de algoritmos y programación de computadoras, además de la 
experimentación incitando al usuario a diseñar, armar y mejorar sus propios diseños de software
y hardware. Que complemente el modelo propuesto por la RIEB, poniendo como actor a la tecnología
en este caso la computadora y sus aplicaciones.



\subsection{Productos Esperados}
Al concluir este proyecto, se espera tener un intérprete del lenguaje de programación
propuesto (Olinki), un entorno de desarrollo integrado(EDI)
con soporte gráfico para el lenguaje de programación Olinki, el cual permita la creación
de programas de computadora de manera lúdica y simple mediante el uso de iconos.
Además de una plataforma de hardware basado en la raspberry pi donde se podrá programar y utilizar el EDI Olinki 
sin necesidad de una computadora externa, así como un sensor de temperatura, luminosidad, y contacto
 conectables a ésta.\\
 De igual manera, contará con un manual de usuario donde vendrá descrito el lenguaje de programación,
 el uso del EDI, así como dos programas de ejemplo con su respectiva contra parte en hardware.
 
 \section{Marco Teórico}
 
 



\end{document}
