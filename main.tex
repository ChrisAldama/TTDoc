\documentclass[letterpaper,10pt]{article}
\usepackage[utf8]{inputenc}
\usepackage[spanish]{babel}
\selectlanguage{spanish}

\title{Mecanismo programable para niños}
\author{Aldama Pérez Cristopher\\ Zavala Ventura Miguel Ángel }
\date{2015}

\begin{document}
\maketitle
\pagenumbering{gobble}
\newpage
\pagenumbering{arabic}

\section{Resumen}

Mecanismo programable para niños es un sistema que consta de un lenguaje gráfico 
de programación y un bloque micontrolador al que se le pueden conectar sensores 
(contacto, luz y temperatura) y actuadores (motores), que sirve como apoyo a la 
enseñanza de programación y robótica a niños de primaria de entre 7 y 11 años.\\

\textbf{Palabras clave}: Lenguaje, Sensor, Actuador, Robótica, Programación, Educación.

\newpage
\section{Advertencia}
\textit{``Este trabajo contiene información desarrollada por la Escuela
 Superior de Cómputo del Instituto Politécnico Nacional a partir de datos y documentos
  con derecho de propiedad y por lo tanto su uso queda restringido a las 
  aplicaciones que explícitamente se convengan.''}

\newpage  
\tableofcontents
\newpage
  
\newpage
\section{Objetivo}

Analizar, implementar y diseñar un sistema de cómputo, tanto en hardware como
en software que permita a niños de entre 7 y 11 años crear programas simples
usando iconos gráficos, así como su ejecución e interacción con sensores y motores,
con la finalidad de ayudar en la enseñanza de programación y uso de computadoras.

\newpage
\section{Introducción}

En el campo de la educación básica, tanto en escuelas públicas como privadas, 
una de las principales preocupaciones es enseñar conceptos relacionados 
con la tecnología, debido a la exposición que la sociedad tiene 
con ella es cada vez mayor, y para ello desarrollan competencias en las 
que motivan el conocimiento, uso y aplicación de la computadora en las tareas 
de la vida diaria; sin embargo, aunque en el mercado existen diversos materiales 
para su enseñanza, es difícil encontrar alguno que mantenga el interés de los niños 
pequeños y se ajuste al ritmo en el que absorben las ideas.
\newline

Actualmente los dispositivos en los que se apoya la enseñanza de estos 
conceptos son circuitos básicos, que están listos para armarse, sin embargo 
limitan la interacción a la observación de su funcionamiento, 
lo cual pierde trascendencia e interés al poco tiempo. También existen 
sistemas más robustos, mecanismos controlados por un programa de computadora, 
para el que se necesita un nivel de abstracción mayor, 
pues requiere de la comprensión de conceptos de matemáticas y lógica.
\newline

La propuesta de este proyecto consiste en crear un dispositivo cuyo 
funcionamiento pueda ser aprendido de forma gradual 
haciendo uso del juego y para ello se debe analizar, 
diseñar, probar e implementar un sistema mecánico programable que sirva como 
material auxiliar en la enseñanza de conceptos, en el área de la lógica
y la programación de sistemas de cómputo, enfocado en niños de escuelas primarias,
con edad de entre 7 y 11 años de edad.
\newpage

\section{Problemática}
En este capítulo se plantea el problema en el que este proyecto se enfoca, así como
determinar los objetivos específicos, la justificación y se en listan los resultados esperados.
\newline

\subsection{Planteamiento del problema}

La tecnología va adquiriendo día a día un lugar más importante en el
desempeño de las tareas diarias, que van desde las compras en el súper mercado,
operaciones bancarias, entretenimiento y actividades lúdicas. La computadora 
y sus aplicaciones tienen un rol central en el desarrollo de la sociedad, 
es por eso que escuelas en especial las de educación primaria
buscan herramientas que ayude a sus alumnos a tener conocimiento adecuado 
sobre las ciencias de la educación.
\newline
Este trabajo terminal, se presenta como una herramienta en





\end{document}
